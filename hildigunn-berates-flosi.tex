Flosi and his band of men have been riding around through the farms,
gathering news about the aftermath of the slaying of Hoskuld. As soon as
they return to Hoskuld's home, Hoskuld's wife Hildigunn prepares an
elaborate feast for Flosi and his band of men. Flosi seems to suggest
that he's insulted by the preparations, saying that there's ``no need to
fix up the high seat for me, and no need to make fun of me.''(116) Flosi
might be suggesting that Hildigunn is showing off Hoskuld's wealth and
importance by putting on this elaborate feast. He lets Hildigunn know
that he realizes what she's doing, and tries to humble himself.

Flosi's intuition serves him well. The first recorded words he hears
from Hildigunn are: ``'What action can I expect from you for the
slaying, and what support'''``(116). Hildigunn has shown without words
that Hoskuld was an important and valuable man. She immediately and
pragmatically moves to consolidate Flosi's supprort. In response, Flosi
pledges less than full support. He only promises to prosecute the case
to the full extent of the law, or obtain an honorable settlement. By
pledging to prosecute the case to the full extent of the law, Flosi is
denying Hildigunn blood vengence. Flosi is familiar with feuds, and he
knows that the pledge of vengence that Hildigunn wants will lead to many
deaths. Flosi wants to avoid promising vengence, for his safety, and the
safety of his men.

As soon as Hildigunn hears only lukewarm support from Flosi, she moves
to emasculate him in an effort to secure his support for vengence
against the slaying of Hoskuld. First, Hildigunn compares Flosi's
manliness to Hoskuld's: ``\,`Hoskuld would have extracted
blood-vengeance if it were his duty to take action for you.'\,''(193).
She then calls Flosi's two brothers manlier than him, saying that they
killed a man for a lesser insult. Hildigunn's assertion that Hoskuld
would have taken blood vengence if he were alive is dubious at best,
since Hoskuld remarkably peaceful. Hoskuld is in fact so peaceful that
he does not resist the Njallssons when they kill him. He even asks God
to forgive them (111). Since anybody who knew Hoskuld would have sincere
doubts that he would ever take blood vengence for anybody, Hildigunn's
insult must not be a comment about Hoskuld. It must be an insult to
Flosi's manhood. When Hildigunn calls Flosi's brothers manlier than he,
she is trying to constrain him within honor culture. Flosi's brothers
killed a man for insulting their father. For Flosi to let this killing
go unpunished would be disgraceful, and would degrade his status in the
community relative to anybody able to exact vengence. It's Hildigunn's
invocation of relative status which goads Flosi the most up to this
point. Flosi still does not promise his unqualified support to exact
blood vengence for Hoskuld.

After calling other man manlier than Flosi, Hildigunn places Hoskuld's
bloody clothes on Flosi. She charges him: ``In the name of God and all
good men I charge you, by all the powers of your Christ and by your
courage and manliness, to avenge all the wounds which he recieved in
dying- or else be an object of contempt to all men.'' Here, Hildigunn
makes explicit some of her earlier, subtler insults. She explicitly
challenges Flosi's manliness, saying that either he will take blood
vengence for Hoskuld, or he is not a man, but a creature deserving of
contempt. By saying ``your Christ'' instead of ``Christ'', Hildigunn
separates her gods from Flosi's. She invokes the powers of Christ in
favor of vengence, implying that failing to exact vengence will
emasculate Christ as well as Flosi. Hildigunn can hardly make any charge
more strong than direct challenge to Flosi's manliess and the manliness
of Flosi's gods. Her challege makes a settlement unmanly, and vengence
the only option.

As soon as Flosi hear Hildigunn's challenge to his manliness, he is
furious. He calls Hildigunn ``the worst monster'' and charges her with
``wanting to take the course that is the worst for us all.''(116) He is
furious because he is aware of how Hildigunn just constrained him.
Because of her goading, Flosi will lost manliness, and lost status, if
he does not avenge Hoskuld. This was Hildigunn's goal, and she used
every weapon at her disposal to force Flosi to take vengence. Flosi
knows that killing the Njalssons will lead to many deaths, including
his. However, we will see that this dim prospect does not stop Flosi
from following the path that Hildigunn sets out for him. Flosi's fury at
Hildigunn is a measure of how constraining her words are. Flosi would
not be furious for nothing. Flosi recognizes that Hildigunn has started
a feud which will lead to many deaths, and he recognizes that his part
is only now to take vengence. Flosi retorts: ``\,`Cold are the councils
of women'\,'', and his face ``was, in turn, as red as blood, as pale as
grass, and as black as Hel itself.''(116). Flosi can see the burning
fast approaching, and because of Hildigunn's goading, he cannot stop it
without sacrificing his honor.
